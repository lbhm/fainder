%!TEX root = ../main.tex

\section{Introduction}
\label{sec:intro}

In today's data-driven world, where organizations collect vast amounts of information from various sources, efficient and effective data discovery has become indispensable.
The increasing importance of data discovery is primarily driven by the growing popularity of machine learning techniques, which require substantial volumes of data.
Consequently, this trend has led to a surge in data sharing and trading within and across organizations~\cite{azcoitia_survey_2022, kennedy_revisiting_2022}.

However, most data discovery systems (and hence data sharing platforms) have limited utility due to two critical design choices that we summarize in Table~\ref{tab:solution_space}.
First, these systems typically assume that all datasets are \emph{completely accessible to the search algorithm} for indexing and processing~\cite{castelo_auctus_2021, castro_fernandez_aurum_2018, galhotra_metam_2023, gong_ver_2023}.
This assumption neglects the distributed nature of data repositories.
Moving data to a central server is often infeasible due to cost reasons and data owners' reluctance to relinquish control over their proprietary datasets.
Instead, a widely adopted data-sharing paradigm involves a distributed collection of data repositories, where each data provider only shares the \emph{metadata} with a search engine.
This is commonly observed in data market platforms, such as Datarade~\cite{datarade_find_2024} and Dawex~\cite{dawex_data_2024}, as well as federated data settings, such as Gaia-X~\cite{braud_road_2021} and Agora~\cite{traub_agora_2021}.
Second, existing systems generally depend on users conducting keyword search along with filters or providing example data to find relevant datasets~\cite{castelo_auctus_2021, noy_google_2019}.
Even though this search mode is intuitive, it limits discovery in scenarios where users have specific data distribution requirements.
For example, users training a machine learning model might seek datasets with a substantial number of samples from each target group to avoid overfitting.
More broadly, lack of access to a representative sample for data analysis (also known as selection bias) leads to flawed and unreliable outcomes.
These challenges are particularly evident in industrial settings due to repurposing or reusing data~\cite{bethlehem_selection_2010, culotta_reducing_2014, greenacre_importance_2016, zhu_consistent_2023}.
Several data science pipelines have failed because of poor representation of training data~\cite{dastin_amazon_2022, mulshine_major_2015, rose_are_2010, townsend_most_2017}.\looseness=-1

\begin{table}[t]
    \caption{Overview of query-driven dataset search approaches, categorized by search mode and data access model.}
    \label{tab:solution_space}
    \centering
    \begin{tabularx}{\linewidth}{>{\raggedright\arraybackslash}X c c}
        \toprule
        & \multicolumn{2}{c}{Data access model} \\
        \cmidrule{2-3}
        Search mode & Full access & Metadata access\\
        \midrule
        Keyword & \cite{castelo_auctus_2021, castro_fernandez_aurum_2018} & \cite{noy_google_2019, open_knowledge_foundation_ckan_2022, zhang_ad_2018} \\
        By example$^\ddagger$ & \cite{bharadwaj_discovering_2021, bogatu_dataset_2020, castelo_auctus_2021, castro_fernandez_aurum_2018, rezig_dice_2021, santos_sketch-based_2022} & - \\
        Distribution-aware & \cite{asudeh_towards_2022, chai_selective_2022, nargesian_tailoring_2021} & \system{} \\
        \bottomrule
    \end{tabularx}
    \raggedright\footnotesize
    $^\ddagger$ This includes finding joinable and unionable tables based on an input table.\\
\end{table}

Together, the full data access assumption and restricted search mode form a critical roadblock in developing practical data discovery systems.
We demonstrate this with the following example.

\begin{example}
Consider a data scientist training a cancer prediction model.
After developing a prototype, they want to test the robustness of their solution with data from similar trials at other hospitals.
To qualify for their work, a dataset must cover different patient ages, so at least 30\% of patients should be younger than 40 and at least 30\% older than 60.
Since such studies contain sensitive information, accessing them requires approval.
Therefore, the datasets are not centrally gathered but hosted by the organizations that own each dataset.
Consequently, the scientist must search through the publicly available metadata of datasets across several independent data repositories.
\end{example}

While the status quo for search over decentralized data provides basic functionality, it fails to address the nuanced complexities of searching datasets with distributional requirements.
In the example above, our medical scientist only has two options when using keyword-based dataset search: (1)~either pose a general keyword query (e.g.,~``cancer'') and manually review a large number of datasets or (2)~add more keywords to the query, hoping that all those keywords are included in the dataset description so that no relevant dataset is filtered out.
As a consequence of these functional limitations, users often face too many, off-topic, or no results at all.


In this work, we study the novel problem of distribution-aware dataset search over decentralized data repositories, which complements existing search paradigms.
To address the problem, we must tackle three main challenges:
(C1)~We must develop easily adoptable methods for searching over decentralized datasets;
(C2)~users must be able to express distributional requirements in search queries; and
(C3)~search engines must identify datasets that satisfy distributional requirements accurately and efficiently at scale.
To address all these challenges, we propose \system{}, an index for distribution-aware data discovery without raw data access, and a new query model.

\paragraph{(C1) Dataset profiles for search over decentralized data.}
Similar to existing data discovery systems, \system{} assumes that data providers share a profile for each of their datasets with a search engine.
To lower the barrier to enriching existing dataset profiles with distribution-aware data synopses, a solution must require as little additional information and effort beyond the status quo as possible.
One of the most widespread and simple yet flexible synopses are histograms~\cite{cormode_synopses_2011}.
They are easy for data owners to generate and seamlessly integrate into dataset profiles.
Some dataset search engines, such as Auctus~\cite{castelo_auctus_2021} or Kaggle~\cite{kaggle_inc_kaggle_2024}, already present histograms as a visual synopsis to the user (but do not use them for distribution-aware search).
\system{} allows each data owner to create histograms of their data independently, as it is robust to heterogeneous histograms.
Thus, they may individually choose the histogram granularity according to their data privacy sensitivity.
If a data owner refuses to provide histograms for a dataset, the search engine can always fall back to only using existing search techniques.\looseness=-1

\paragraph{(C2) Percentile predicates for specifying user requirements.}
We introduce a new type of search predicate that we call \emph{percentile predicate} to offer a simple and intuitive way of specifying distributional requirements.
Abstractly, a percentile predicate requires that the dataset values from a given range must or must not represent more than a certain percentage of all values.
When composing multiple predicates, this allows users to approximate entire statistical distributions, such as a normal distribution.
To integrate with prior work, we propose a simple query model based on Boolean algebra that enables searching for datasets by seamlessly combining existing keyword-based techniques with distributional requirements.

\paragraph{(C3) Indexing for accurate and efficient dataset search.}
Designing an index for dataset search over decentralized data repositories requires taking multiple stakeholders into account: data owners provide heterogeneous histograms, users expect accurate query results at interactive response times, and search engines aim to scale to extensive dataset collections with minimal resource footprint.
We present two variants of \system{} to overcome this challenge.
\approximate{} optimizes the execution time and allows users to search with full precision or recall guarantees.
\exact{} combines these guarantees in a multi-step solution to prune the search space efficiently.
Both \system{} variants address histogram heterogeneity by transforming the unique bins of independently generated histograms into a globally aligned bin distribution.
Leveraging the aligned bins, \system{} uses binary search at query time to navigate the search space efficiently.
Furthermore, we employ clustering to optimize the trade-off between index accuracy and size.\looseness=-1

\paragraph{Outline of Contributions.}
After introducing basic concepts and our notation in Section~\ref{sec:preliminaries}, we make four major contributions.

\begin{itemize}[left=0pt]
    \item
    We formally define the problem of distribution-aware dataset search on decentralized data repositories and propose a minimalistic yet effective query model that combines existing search techniques (e.g.,~keywords) and percentile predicates (Section~\ref{sec:problem}).

    \item
    We present \system{} (Section~\ref{sec:index_overview}), an index for percentile predicates that can be constructed on collections of heterogeneous, independently generated histograms (Section~\ref{sec:index_construction}).

    \item
    We introduce two query modes that trade off runtime and result accuracy (Section~\ref{sec:index_querying}).
    \approximate{} achieves sublinear scaling in the number of datasets while offering different trade-offs for precision and recall.
    \exact{} yields exact query results whilst being significantly faster than the state-of-the-art.

    \item
    We conduct an extensive experimental evaluation on real-world open dataset collections (Section~\ref{sec:evaluation}).
    Our evaluation shows that \approximate{} is up to more than two orders of magnitude faster than our baselines.
    In addition, \exact{} also is up to $25\times$ faster than the state-of-the-art.
\end{itemize}

We close our study with a review of related work in Section~\ref{sec:related_work} and an outlook on future research directions in Section~\ref{sec:conclusion}.
In short, we are the first to investigate distribution-aware dataset search over decentralized data repositories, which is a critical step towards making dataset search practical.
